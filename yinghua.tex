%-----------------------------------------
% Yinghua Wu
% URL: https://github.com/yhwu/resume.git
%-----------------------------------------

\documentclass[11pt,letterpaper,final]{moderncv}
\usepackage{fontspec}
\moderncvtheme[blue]{classic} %blue, orange, green, red, purple, grey, Casual
\usepackage{url}

% DOCUMENT LAYOUT
\usepackage[scale=0.85]{geometry}
\setlength{\hintscolumnwidth}{2.5cm}
\AtBeginDocument{\recomputelengths}
\usepackage[utf8]{inputenc}
\defaultfontfeatures{Mapping=tex-text} % converts LaTeX specials (``quotes'' --- dashes etc.) to unicode


% Personal Information 
\firstname{Yinghua}
\familyname{Wu}
\address{3610 Hamilton Walk}{Philadelphia, PA 19104}
\mobile{(610) 572-2436}
\email{yinghua@upenn.edu}
\extrainfo{\url{https://github.com/yhwu}}


\title{Yinghua Wu's Resume}

\nopagenumbers{}

\begin{document}

\maketitle


\section{Summary}
\cventry{}{Eligible to work in US}{}{}{} {
  \begin{itemize}
  \item I have made fundamental contributions in quantum dynamics research. I have strong mathematical abilities and keen intuition.
  \item I have developed statistical analysis methods and pipelines for various cancer treatment studies and gene therapy trials. I have strong numerical and project management skills.
  \item My next career objective is to develop predicting algorithms through deep learning and build self updating platforms to constantly improve them.
  \end{itemize}
}


\section{Experience} 
\cventry{Oct 2014 -- Present}{Senior Research Investigator}{University of Pennsylvania}{Dept. of Microbiology}{} {
  \begin{itemize}
  \item Implemented N-mixture model to estimate number of integrated cells in gene therapy cancer treatments.
  \item Developed efficient clustering algorithm to classify vector integrated cells.
  \item Developed highly efficient analysis pipeline based on next generation sequencing for gene therapy trials including Was-Lenti, beta-Thal, SCID1, SCID2, and most recently, CART19 trials.
  \item Developed analysis pipeline for Dam-ID protocol for HIV genome integration.
  \end{itemize}
}


\cventry{Jun 2011 -- Sep 2014}{Postdoctoral Research Fellow}{University of Pennsylvania}{Dept. of Biostatistics}{}{
  \begin{itemize}
  \item Developed an algorithm to evaluate parent of origin effect based on maximum likelihood ratio test and haplotype frequencies.
  \item Developed robust segment identification method based on negative binomial transformation for copy number variation detection.
  \item Developed a copy number variation detection algorithm based on CIGAR string, edit distance, and abnormal pair length.
  \item Developed permutation method to control batch effect for RNAseq gene expression studies. 
  \item Performed genome-wide association studies for eye diseases and identified rare coding SNPs that might be causal factors for long/short sightedness.
  \end{itemize}
}


\cventry{Aug 2010 -- May 2011}{Programmer analyst}{Collaborative Center for Statistics in Science}{New Haven, CT}{}{
  \begin{itemize}
  \item  Developed a coherent sampling algorithm for genotype/phenotype simulations that retains linkage disequilibrium structures for all variants including rare ones.
  \item Developed imputation method for rare variants based on associations for whole genome studies.
  \end{itemize}
}


\cventry{May 2007 -- Dec 2008}{Postdoctoral Research Associate}{Georgia Institute of Technology}{Dept. of Chemistry}{}{
  \begin{itemize}
  \item Derived a closed form expression for the auto-correlation function for large organic molecules under Duschinsky rotation effect.
  \item Derived a closed form expression for the photo emission spectrum for large organic LED molecules based on Feynman path integral theory.
  \end{itemize}
}


\cventry{Jun 2004 -- May 2007}{Postdoctoral Research Associate}{Tulane University}{Dept. of Chemistry}{}{
  \begin{itemize}
  \item Developed and theoretically proved the time-dependent semi-classical two-state surface hopping method, which is accurate to the first order of the Planck constant.
  \item Proved mathematical lemma that the stationary phase approximation to the Feynman path integral is accurate to the first order of the Planck constant or the small number.
  \end{itemize}
}


\section{Education}
\cventry{1999 -- 2004}{PHD}{Yale University}{Theoretical Physical Chemistry}{}{
  \begin{itemize}
  \item Developed a two-state quantum dynamical propagator for multi-dimensional systems. 
  \item Developed the matching pursuit split operator Fourier transform method for multi-dimensional quantum dynamical wave function evolution.
  \end{itemize}
}
\cventry{1993 -- 1998}{BS}{University of Science and Technology of China}{Chemistry}{}{}


\section{Languages}
\cvlanguage{English}{Professional}{}
\cvlanguage{Chinese}{Native}{}

\section{Skill sets}
\cvline{Computer}{\small Comprehensive: C, C++, Fortran, PERL, R, SQL}
\cvline{}{\small Functional: Python, PHP}
\cvline{Bioinformatics}{\small Full stack}
\cvline{Biostatistics}{\small Advanced}
\cvline{Math}{\small Advanced}
\cvline{Miscellaneous}{\small Windows, Office, Linux, HPC}

\end{document}


